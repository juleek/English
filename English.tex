\documentclass[10pt,a4paper]{article}
\usepackage[utf8]{inputenc}
\usepackage[english,russian]{babel}
\usepackage{indentfirst}
\usepackage{multicol}
\usepackage[left=10mm, right=10mm, top=10mm, bottom=10mm]{geometry}
\usepackage[normalem]{ulem}

%\usepackage{setspace}
%\setstretch{0.8}

\pagenumbering{gobble}

\usepackage{enumitem} % Do not indent lists and for
\setlist{nosep}


\newlength{\OriginalParIndent}
\OriginalParIndent=\the\parindent % Remember original parindent
%\showthe\parindent
%\showthe\tmp
\setlength{\parindent}{0pt} % Do not indent every paragraph



\renewcommand{\baselinestretch}{0.97}
\newcommand\ex[1]{\textit{\textbf{{#1}}}}
\newcommand\sex[1]{\textit{\textbf{{\small {#1}}}}}

\begin{document}

%================================================================================================
%================================================================================================
%================================================================================================
\setcounter{secnumdepth}{1} % Enumerate only Sections (1) and SubSecions (2)
\twocolumn[\section{SpeakOut Intermediate}]
\subsection{Unit 1}
\subsubsection{Idioms}

\begin{description}[leftmargin=5cm,style=nextline,before={\renewcommand\makelabel[1]{##1 ~---}}]
\item[\ex{Take part in something}] принять участие в чём-то
\item[\ex{Take after someone}] родиться похожим на
\item[\ex{Take responsability for}] брать ответсвенность
\item[\ex{Get on with someone}] ладить с
\item[\ex{Do someone a favor}] сделать одолжение
%\item[\ex{Something}] Text. More text.More text.More text. More text. More text.
%                      More text. More text. More text. More text. More text. More text.
%                      More text.
\end{description}



%================================================================================================
%================================================================================================
%================================================================================================
\subsection{Unit 2}
\subsubsection{Preposition + expressions of time}
To help you remember which preposition of time to use, try to memorise this: \textbf{on} M\textbf{on}day; \textbf{in} w\textbf{in}ter; \textbf{at} th\textbf{at} time.

\begin{description}[leftmargin=1.2cm,style=nextline,before={\renewcommand\makelabel[1]{##1 ~=}}]
\item[\textbf{on}] for specific days
\item[\textbf{in}] for time periods
\item[\textbf{at}] for specific times
\end{description}


\begin{description}[leftmargin=2.8cm,style=nextline,before={\renewcommand\makelabel[1]{##1 ~---}}]
\item[\ex{On my own}] alone, not whith other people
\item[\ex{On purpose}] it was not mistake
\item[\ex{In a hurry}] cannot wait
\end{description}


\subsubsection{Vocabulary}
\begin{description}[leftmargin=2.3cm,style=nextline,before={\renewcommand\makelabel[1]{##1 ~---}}]
\item[\ex{Violent}] насильственный
\item[\ex{Strike}] забастовка, удар
\item[\ex{Fugitive}] беглец
\item[\ex{Hostages}] заложники
\item[\ex{Prisoner}] заключённый
\end{description}



%================================================================================================
%================================================================================================
%================================================================================================
\subsection{Unit 3}
\subsubsection{Idioms}
\begin{description}[leftmargin=4.5cm,style=nextline,before={\renewcommand\makelabel[1]{##1 ~---}}]
\item[\ex{We are running out of time}] время на исходе
\end{description}

\begin{description}[leftmargin=4.8cm,style=nextline,before={\renewcommand\makelabel[1]{##1 ~---}}]
%\itemsep -0.2em
\item[\ex{Go window shopping}] прогуливаться по магазинам, разглядывать ветрины
\item[\ex{Let your hair down}] позволить себе вести себя более спокойно и расслаблено
\item[\ex{Be in two minds}] колебаться, не знать на что решиться
\item[\ex{Work against the clock}] работать не покладая рук
\item[\ex{It's not my cup of tea}] это не в моём вкусе
\item[\ex{Close to my heart}] близко моему сердцу
\item[\ex{Put my foot in it}] дать маху
\end{description}

\begin{description}[leftmargin=3.3cm,style=nextline,before={\renewcommand\makelabel[1]{##1 ~---}}]
\item[\ex{Break the ice}] положить начало
\item[\ex{Learn by heart}] учить на изусть
\item[\ex{Travel light}] путешествовать на-легке
\item[\ex{Go grey}] седеть
\item[\ex{Small talk}] болтовня
\item[\ex{On our mind}] на уме
\item[\ex{In hot water}] в беде (по своей вине)
\end{description}



 %================================================================================================
 %================================================================================================
%================================================================================================
\subsection{Unit 4}
\subsubsection{Confusing words}

\begingroup
\parindent=\OriginalParIndent

\ex{job~--- work}. \textbf{Work} is what you do to earn money: \textit{What kind of work does he do?}
A \textbf{job} is the particular type of work that you do: \textit{Sam's got a job as a waiter}.
\textbf{Job} can be plural, but \textbf{work} cannot.

\ex{remember~--- remind}. If you \textbf{remember} something, a fact or event from the past,
or something you earlier decided to do, comes back into you mind:
\textit{He suddenly remembered he had to go to the bank}. If someone \textbf{reminds} you to
do something, or something reminds you of something, they make you remember it:
\textit{Remind me to call him later today}.


\ex{forget~--- leave}. If you want to talk about the place where you have left something, use the
verb \textbf{leave}, not the verb \textbf{forget}. Compare: \textit{I've forgotten my book and I've
forgotten my keys.} \textit{I've left my keys in the car}.\\
Don't say: \textit{\sout{I've forgotten my keys in the car.}}

\ex{hear~--- listen}.

\ex{earn~--- win}.

\ex{fun~--- funny}. Use \textbf{fun} to talk about events and activities that are enjoyable, such as
games and parties: \textit{Let's go to the beach and have some fun.} \textbf{Funny} is an adjective
that describes someone or something that makes you laugh: \textit{Bob's jokes are really funny.}

\ex{miss~--- lose}. \textbf{miss}~--- пропустить; \textbf{lose}~--- потерять.

\endgroup


\subsubsection{Vocabulary: VERY -> withour VERY}
\begingroup
\setlength{\columnsep}{19pt}
\begin{multicols}{2}
\begin{description}[leftmargin=1.6cm,style=nextline,before={\renewcommand\makelabel[1]{##1 ~---}}]
  \item[good] wonderful, amazing, brilliant
  \item[big] enormous, giant
  \item[small] tiny
  \item[tired] exhausted
  \item[hot] boiling
  \item[cold] freezing
  \item[tasty] delicious
  \item[angry] furious
  \item[pretty] beautyful
\end{description}

\begin{description}[leftmargin=2.3cm,style=nextline,before={\renewcommand\makelabel[1]{##1 ~---}}]
  \item[interesting] fascinating
  \item[difficult] impossible
\end{description}
\end{multicols}
\endgroup



\subsubsection{Covering letter}

\begin{description}[leftmargin=\OriginalParIndent,style=nextline,before={\renewcommand\makelabel[1]{##1 ~---}}]
  \item[\ex{Regarding your advertisement\dots}] about
  \item[\ex{I would like to submit an application for the\dots}] to apply for the job
  \item[\ex{I meet all the requirements\dots}] I think I would be good for the job
  \item[\ex{Proven ability in\dots}] I have shown that I am able to do this
  \item[\ex{\dots to hearing from you at your earliest convenience}] as soon as you have the opportunity
\end{description}












%================================================================================================
%================================================================================================
%================================================================================================

\newpage
\setcounter{secnumdepth}{2} % Enumerate only Sections (1) and SubSecions (2)
\twocolumn[\section{The United Kingdom. 2016.}]

\subsection{Phrases from speech}
\begin{description}[leftmargin=\OriginalParIndent,style=nextline,before={\renewcommand\makelabel[1]{##1 ~--}}]
\item[] Do many people here \sout{ride} \textbf{commute / travel} to London?
\item[] We drink \sout{much} \textbf{a lot of} tea.
\item[] \sout{Two times in a week} \textbf{Twice} a week.
\item[] \sout{Previous year.} \textbf{Last} year.
\item[] We can show you one \sout{our} \textbf{of ours} videos.
\item[] You \sout{may not} \textbf{doesn't (need / have) to} search.
\item[] I'll make two \sout{circles} \textbf{circuits} around the church.
\item[] A car was the only \sout{mean} \textbf{means}, you could use to get out of country.
\item[] \sout{Have he married?} \textbf{Did he get married?}
\item[] Zebra is an animal, that has black and white \sout{lanes} \textbf{stripes} on its body.
\item[] As soon as we \sout{entered} \textbf{got on} the train.
\item[] We were impressed \sout{that there was a very quiet train} \textbf{by
        how quiet the train was}.
\item[] Which one? \sout{Second one.} \textbf{Later (formerly).}
\item[] Finally, we \sout{exited on} \textbf{traveled to} Westminster station.
\item[] Сделать мэйкап: \sout{Make make-up.} \textbf{Do make-up. / Put on make-up.}
\item[] There are two \sout{parts} \textbf{groups} of people in Korolev.
\item[] Подстричь газон: \sout{Cut the grass.} \textbf{Mow the lawn}.
\item[] She wears a coat to \sout{protect herself from cold} \textbf{keep warm}.
\item[] \sout{Ask help. Ask to help.} \textbf{Ask for help}.

\item[] That would work as well.
\item[] Exam: Pass an exam. Fail an exam. Do well / badly in an exam.
\item[] Take the second right. Excuse me, please, could you (give me directions) /
        (tell me the way) to The London Eye?
\item[] Не могли бы вы поставить (тарелку, например. За столом): Could you please, put it down.
\item[] Ask something in caffee: Please, could I have a green tea? I would like a green tea, please.
\item[] Она подсадила меня на YouTube: She has got me addicted to YouTube.
\end{description}


%================================================================================================
\subsection{Pronunciation}
\begin{description}[leftmargin=\OriginalParIndent,style=nextline,before={\renewcommand\makelabel[1]{##1 ~--}}]
\item[] \textbf{-ng}: ki\textbf{ng}, swimmi\textbf{ng}, includi\textbf{ng}.
\item[] s\textbf{i}x: и.
\item[] \textbf{h}im, \textbf{h}er, \textbf{h}is, \textbf{h}ave.
\item[] \textbf{w}ere: \sout{в}[еэ]. \textbf{у[вф]}[еэ]
\end{description}



%================================================================================================
\subsection{Vocabulary}
\subsubsection{New words, phrases}
\begin{description}[leftmargin=\OriginalParIndent,style=nextline,before={\renewcommand\makelabel[1]{##1 ~---}}]
\item[Allotment] брит. небольшой участок земли, сдаваемый в аренду под огород.
\item[White collar] tv series.
\item[Kettle] чайник.
\item[Teabag] чайный пакетик.
\item[Hedge] живая изгородь.
\item[Hen] курица.
\item[Gravy] подливка (из сока жаркого), соус, сок.
\item[Admit one] ticket for one.
\item[Stroll around] wander.
\item[Afterward, beforehand] не надо ставить после них существительного (в отличие от after, before),
                             например: What will you do afterwards?
\item[Riddle] загадка, тайна.
\item[Tame (squirrels)] приручённый, одомашненный; укрощённый (о животных).
\item[Wages] зарплата.
\item[Renovate] обновленная кухня, реконструированный парк - renovated kitchen, renovated park.
\item[Do a chore] делать повседневную работу по дому.
\end{description}

\subsubsection{Phrases verbs}
\begin{description}[leftmargin=2.2cm,style=nextline,before={\renewcommand\makelabel[1]{##1 ~---}}]
\item[Call off] cancel (a meeting).
\item[Use up] израсходовать.
\item[Clean out] ``обчистить''.
\item[Put on] надеть.
\item[Take off] раздеться.
\item[Make up] invent, выдумать.
\item[Leave out] skip, пропускать, не принимать во внимание.
\item[Lock in] запереть (себя внутри).
\item[Bring up] поднимать вопрос (неприятный).
\item[Hand out] раздавать.
\item[Hand in] передавать (в руки).
\item[Turn over] переворачивать.
\end{description}

\subsubsection{Phrases verbs}
\begin{description}[leftmargin=\OriginalParIndent,style=nextline,before={\renewcommand\makelabel[1]{##1 ~---}}]
\item[Bad hair day] a day on which everything seems to go wrong.
\item[Can't stand / Can't bear] means to be unable to tolerate somebody or something.
\end{description}

\subsubsection{Houses / Flats}
\begin{description}[leftmargin=\OriginalParIndent,style=nextline,before={\renewcommand\makelabel[1]{##1 ~---}}]
\item[Flat] квартира в многоквартирном доме.
\item[Apartment building] многоквартирные дом.
\item[House] частный дом.
\end{description}

\subsubsection{Go/come, take/bring}
todo

\subsubsection{Degree of likability}

\begin{itemize}
  \item Despise~--- презирать.
  \item Abhor / Detest~--- ненавидеть, питать отвращение; не выносить, не терпеть.
  \item Hate
  \item Dislike / don't like
  \item To be easy: I am easy (informal).
  \item Do not mind~-- it's ok. Examples:
  \begin{itemize}
    \item Do you mind V\textsubscript{ing}? No, I don't mind.
    \item Do you mind helping me with my homework? I don't mind helping you with…
  \end{itemize}
  \item Quite like / quite enjoy.
  \item Like / enjoy.
  \item Love.
  \item Adore.
\end{itemize}


%================================================================================================
\subsection{Riddle}
Smith, where Jones had had ``had'', had had ``had had''. ``Had had'' had had examiner's approval.




%================================================================================================
\subsection{Pronouns}
I don’t like vegetables, I don't like \sout{it} \textbf{them}.


\end{document}
