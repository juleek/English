\documentclass[10pt,a4paper]{article}
\usepackage[utf8]{inputenc}
\usepackage[english,russian]{babel}
\usepackage{indentfirst}
\usepackage{multicol}
\usepackage[left=10mm, right=10mm, top=15mm, bottom=15mm]{geometry}
\usepackage[normalem]{ulem}

%\usepackage{setspace}
%\setstretch{0.8}

\usepackage{enumitem}
\setlist{nosep}

\renewcommand{\baselinestretch}{0.97}
\newcommand\ex[1]{\textit{\textbf{{#1}}}}
\newcommand\sex[1]{\textit{\textbf{{\small {#1}}}}}

\begin{document}
\twocolumn[]


\section{Unit 1}
\textbf{\underline{Idioms}}

\begin{enumerate}
%\itemsep -0.2em
  \item \ex{Take part in something}~--- принять участие в чём-то
  \item \ex{Take after someone}~--- родиться похожим на
  \item \ex{Take responsability for}~--- брать ответсвенность
  \item \ex{Get on with someone}~---ладить с
  \item \ex{Do someone a favor}~--- сделать одолжение
\end{enumerate}




\section{Unit 2}

\textbf{\underline{Preposition + expressions of time}}

To help you remember which preposition of time to use, try to memorise this: \textbf{on} M\textbf{on}day; \textbf{in} w\textbf{in}ter; \textbf{at} th\textbf{at} time. 

\textbf{on} = for specific days

\textbf{in} = for time periods

\textbf{at} = for specific times

\begin{enumerate}
  \item \ex{On my own}~--- alone, not whith other people
  \item \ex{On purpose}~--- it was not mistake
  \item \ex{In a hurry}~--- cannot wait
\end{enumerate}

\textbf{\underline{Vocabulary}}

\ex{Violent}~-- насильственный

\ex{Strike}~-- удар

\ex{Fugitive}~-- беглец

\ex{Hostages}~-- заложники

\ex{Prisoner}~-- заключённый





\section{Unit 3}
\textbf{\underline{Idioms}}

\begin{enumerate}
%\itemsep -0.2em
  \item \ex{Break the ice}~--- положить начало 
  \item \ex{Learn by heart}~--- учить на изусть
  \item \ex{Go window shopping}~--- прогуливаться по магазинам, разглядывать ветрины
  \item \ex{Travel light}~--- путешествовать на-легке
  \item \ex{Let your hair down}~--- позволить себе вести себя более спокойно и расслаблено
  \item \ex{Be in two minds}~--- колебаться, не знать на что решиться
  \item \ex{Go grey}~--- седеть
  \item \ex{Work against the clock}~--- работать не покладая рук
  \item \ex{Small talk}~--- болтовня
  \item \ex{On our mind}~--- на уме
  \item \ex{It's not my cup of tea}~--- это не в моём вкусе
  \item \ex{Close to my heart}~--- близко моему сердцу
  \item \ex{In hot water}~--- в беде (по своей вине)
  \item \ex{Put my foot in it}~--- дать маху
  \item \ex{We are running out of time}~--- время на исходе
\end{enumerate}

 
 


\section{Unit 4}

\textbf{\underline{Confusing words}}
\begin{enumerate}
  \item \ex{job~--- work}
  
  \textbf{Work} is what you do to earn money: \textit{What kind of work does he do?} A \textbf{job} is the particular type of work that you do: \textit{Sam's got a job as a waiter}. \textbf{Job} can be plural, but \textbf{work} cannot.
  \item \ex{remember~--- remind}
  
  If you \textbf{remember} something, a fact or event from the past, or something you earlier decided to do, comes back into you mind: \textit{He suddenly remembered he had to go to the bank}. If someone \textbf{reminds} you to do something, or something reminds you of something, they make you remember it: \textit{Remind me to call him later today}.
  \item \ex{forget~--- leave}
  
  If you want to talk about the place where you have left something, use the verb \textbf{leave}, not the verb \textbf{forget}. Compare: \textit{I've forgotten my book and I've forgotten my keys.} \textit{I've left my keys in the car}. 
  
  Don't say: \textit{\sout{I've forgotten my keys in the car.}}
  \item \ex{hear~--- listen}
  \item \ex{fun~--- funny}
  
  Use \textbf{fun} to talk about events and activities that are enjoyable, such as games and parties: \textit{Let's go to the beach and have some fun.} \textbf{Funny} is an adjective that describes someone or something that makes you laught: \textit{Bob's jokes are really funny.}
  \item \ex{miss~--- lose}
  
  \textbf{miss}~-- пропустить
  \textbf{lose}~-- потерять
  \item \ex{earn~--- win}
  
\end{enumerate}

\par\medskip\textbf{\underline{Covering letter}}
\begin{enumerate}
  \item \ex{Regarding your advertisement\dots}~--- about
  \item \ex{I would like to submitan application for the\dots}~--- to apply for the job
  \item \ex{I meet all the requirements\dots}~--- I think I would be good for the job
  \item \ex{Proven ability in\dots}~--- I have shown that I am able to do this
  \item \ex{\dots to hearing from you at your earliest convenience}~--- as soon as you have the opportunity
\end{enumerate}

\end{document}
