\documentclass[10pt,a4paper]{article}
\usepackage[utf8]{inputenc}
\usepackage[english,russian]{babel}
\usepackage[left=10mm, right=10mm, top=10mm, bottom=10mm, showframe]{geometry}

% DO indent first paragraph
\usepackage{indentfirst}

% Two-columns
\usepackage{multicol}

% Cells spanning several row in tables
\usepackage{multirow}


\usepackage[normalem]{ulem}

% Disable page numbering
\pagenumbering{gobble}

% Do not indent lists and for
\usepackage{enumitem}
\setlist{nosep}

% ==========================================
% Store original par indent for future uses
% ==========================================
\newlength{\OriginalParIndent}
\OriginalParIndent=\the\parindent % Remember original parindent
%\showthe\parindent
%\showthe\tmp
% ==========================================

% Paragraph tweaks
\setlength{\parindent}{0pt} % Do not indent every paragraph
\setlength{\parskip}{1.1mm}   % More space between paragraphs

% Less space between lines
\renewcommand{\baselinestretch}{0.97}

% Domain specific commands
\newcommand\ex[1]{\textit{\textbf{{#1}}}}           % Example
\newcommand\sex[1]{\textit{\textbf{{\small {#1}}}}} % Small Example
\newcommand\za{\_\thinspace }                       % Zero Article


% Itemize with less indentation
\newenvironment{ItemizeWithOrigParIndent}
    {\begin{itemize}[leftmargin=\OriginalParIndent]}
    {\end{itemize}}


%\newenvironment{NextLineDescription}[1]
%    {\begin{description}[#1,style=nextline,before={\renewcommand\makelabel[1]{##1~---}}]}
%    {\end{description}}


% ==========================================
% Show two column boundaries
% ==========================================
\usepackage{etoolbox}
\newlength\Fcolumnseprule
\setlength\Fcolumnseprule{0.4pt}
\makeatletter
\newcommand\ShowInterColumnFrame{
\patchcmd{\@outputdblcol}
  {{\normalcolor\vrule \@width\columnseprule}}
  {\vrule \@width\Fcolumnseprule\hfil
    {\normalcolor\vrule \@width\columnseprule}
    \hfil\vrule \@width\Fcolumnseprule
  }
  {}
  {}
}
\makeatother
\ShowInterColumnFrame
% ==========================================


\begin{document}

% Thinner bullet in lists see
% http://tex.stackexchange.com/questions/8510/reduce-size-of-bullet-character-in-lists
\renewcommand{\labelitemi}{$\vcenter{\hbox{\tiny$\bullet$}}$}


%================================================================================================
%================================================================================================
%================================================================================================
\setcounter{secnumdepth}{1} % Enumerate only Sections (1) and SubSecions (2)
\twocolumn[\section{SpeakOut Intermediate}]

\subsection{Unit 1}

\subsubsection{Idioms}
\begin{description}[leftmargin=5cm,style=nextline,before={\renewcommand\makelabel[1]{##1~---}}]
  \item[\ex{Take part in something}] принять участие в чём-то
  \item[\ex{Take after someone}] родиться похожим на
  \item[\ex{Take responsability for}] брать ответсвенность
  \item[\ex{Get on with someone}] ладить с
  \item[\ex{Do someone a favor}] сделать одолжение
\end{description}



%================================================================================================
%================================================================================================
%================================================================================================
\subsection{Unit 2}
\subsubsection{Preposition + expressions of time}
To help you remember which preposition of time to use, try to memorise this: \textbf{on} M\textbf{on}day; \textbf{in} w\textbf{in}ter; \textbf{at} th\textbf{at} time.

\begin{description}[leftmargin=1.2cm,style=nextline,before={\renewcommand\makelabel[1]{##1~=}}]
  \item[\textbf{on}] for specific days,
  \item[\textbf{in}] for time periods,
  \item[\textbf{at}] for specific times.
\end{description}

\begin{description}[leftmargin=2.8cm,style=nextline,before={\renewcommand\makelabel[1]{##1~---}}]
  \item[\ex{On my own}] alone, not whith other people.
  \item[\ex{On purpose}] it was not mistake.
  \item[\ex{In a hurry}] cannot wait.
\end{description}


\subsubsection{Vocabulary}
\begin{description}[leftmargin=2.3cm,style=nextline,before={\renewcommand\makelabel[1]{##1~---}}]
  \item[\ex{Violent}] насильственный.
  \item[\ex{Strike}] забастовка, удар.
  \item[\ex{Fugitive}] беглец.
  \item[\ex{Hostages}] заложники.
  \item[\ex{Prisoner}] заключённый.
\end{description}



%================================================================================================
%================================================================================================
%================================================================================================
\subsection{Unit 3}
\subsubsection{Idioms}
\begin{description}[leftmargin=4.5cm,style=nextline,before={\renewcommand\makelabel[1]{##1~---}}]
  \item[\ex{We are running out of time}] время на исходе.
\end{description}
\vspace{-\parskip}
\begin{description}[leftmargin=4.8cm,style=nextline,before={\renewcommand\makelabel[1]{##1~---}}]
  \item[\ex{Go window shopping}] прогуливаться по магазинам, разглядывать ветрины.
  \item[\ex{Let your hair down}] позволить себе вести себя более спокойно и расслаблено.
  \item[\ex{Be in two minds}] колебаться, не знать на что решиться.
  \item[\ex{Work against the clock}] работать не покладая рук.
  \item[\ex{It's not my cup of tea}] это не в моём вкусе.
  \item[\ex{Close to my heart}] близко моему сердцу.
  \item[\ex{Put my foot in it}] дать маху.
\end{description}
\vspace{-\parskip}
\begin{description}[leftmargin=3.3cm,style=nextline,before={\renewcommand\makelabel[1]{##1~---}}]
  \item[\ex{Break the ice}] положить начало.
  \item[\ex{Learn by heart}] учить на изусть.
  \item[\ex{Travel light}] путешествовать на-легке.
  \item[\ex{Go grey}] седеть.
  \item[\ex{Small talk}] болтовня.
  \item[\ex{On our mind}] на уме.
  \item[\ex{In hot water}] в беде (по своей вине).
\end{description}



 %================================================================================================
 %================================================================================================
%================================================================================================
\subsection{Unit 4}
\subsubsection{Confusing words}

\begingroup
\parindent=\OriginalParIndent

\ex{job~--- work}. \textbf{Work} is what you do to earn money: \textit{What kind of \ex{work} does he do?}
A \textbf{job} is the particular type of work that you do: \textit{Sam's got a \ex{job} as a waiter}.
\textbf{Job} can be plural, but \textbf{work} cannot.

\ex{remember~--- remind}. If you \textbf{remember} something, a fact or event from the past,
or something you earlier decided to do, comes back into you mind:
\textit{He suddenly \ex{remembered} he had to go to the bank}. If someone \textbf{reminds} you to
do something, or something reminds you of something, they make you remember it:
\textit{\ex{Remind} me to call him later today}.


\ex{forget~--- leave}. If you want to talk about the place where you have left something, use the
verb \textbf{leave}, not the verb \textbf{forget}. Compare: \textit{I've \ex{forgotten} my book and I've
\ex{forgotten} my keys.} \textit{I've \ex{left} my keys in the car}.\\
Don't say: \textit{\sout{I've forgotten my keys in the car.}}

\ex{hear~--- listen}.

\ex{earn~--- win}.

\ex{fun~--- funny}. Use \textbf{fun} to talk about events and activities that are enjoyable, such as
games and parties: \textit{Let's go to the beach and have some \ex{fun}.} \textbf{Funny} is an adjective
that describes someone or something that makes you laugh: \textit{Bob's jokes are really \ex{funny}.}

\ex{miss~--- lose}. \textbf{miss}~--- пропустить; \textbf{lose}~--- потерять.

\endgroup


\subsubsection{Vocabulary: VERY --> without VERY}
\begingroup
\setlength{\columnsep}{19pt}
\begin{multicols}{2}
\begin{description}[leftmargin=1.6cm,style=nextline,before={\renewcommand\makelabel[1]{##1~---}}]
  \item[good] wonderful, amazing, brilliant
  \item[big] enormous, giant
  \item[small] tiny
  \item[tired] exhausted
  \item[hot] boiling
  \item[cold] freezing
  \item[tasty] delicious
  \item[angry] furious
  \item[pretty] beautyful
\end{description}
\vspace{-\parskip}
\begin{description}[leftmargin=2.3cm,style=nextline,before={\renewcommand\makelabel[1]{##1~---}}]
  \item[interesting] fascinating
  \item[difficult] impossible
\end{description}
\end{multicols}
\endgroup



\subsubsection{Covering letter}
\begin{description}[leftmargin=\OriginalParIndent,style=nextline,before={\renewcommand\makelabel[1]{##1~---}}]
  \item[\ex{Regarding your advertisement\dots}] about
  \item[\ex{I would like to submit an application for the\dots}] to apply for the job
  \item[\ex{I meet all the requirements\dots}] I think I would be good for the job
  \item[\ex{Proven ability in\dots}] I have shown that I am able to do this
  \item[\ex{\dots to hearing from you at your earliest convenience}] as soon as you have the opportunity
\end{description}












%================================================================================================
%================================================================================================
%================================================================================================

\newpage
\setcounter{secnumdepth}{3} % Enumerate only Sections (1) and SubSecions (2)

\twocolumn[\section{The United Kingdom. 2016.}]

\subsection{Phrases from speech}

\begin{ItemizeWithOrigParIndent}
  \item Do many people here \sout{ride} \textbf{commute/travel} to London?
  \item We drink \sout{much} \textbf{a lot of} tea.
  \item \sout{Two times in a week} \textbf{Twice} a week.
  \item \sout{Previous year.} \textbf{Last} year.
  \item We can show you one \sout{our} \textbf{of ours} videos.
  \item You \sout{may not} \textbf{doesn't (need/have) to} search.
  \item I'll make two \sout{circles} \textbf{circuits} around the church.
  \item A car was the only \sout{mean} \textbf{means}, you could use to get out of country.
  \item \sout{Have he married?} \textbf{Did he get married?}
  \item Zebra is an animal, that has black and white \sout{lanes} \textbf{stripes} on its body.
  \item As soon as we \sout{entered} \textbf{got on} the train.
  \item We were impressed \sout{that there was a very quiet train} \textbf{by
        how quiet the train was}.
  \item Which one? \sout{Second one.} \textbf{Later (formerly).}
  \item Finally, we \sout{exited on} \textbf{traveled to} Westminster station.
  \item Сделать мэйкап: \sout{Make make-up.} \textbf{Do make-up. / Put on make-up.}
  \item There are two \sout{parts} \textbf{groups} of people in Korolev.
  \item Подстричь газон: \sout{Cut the grass.} \textbf{Mow the lawn}.
  \item She wears a coat to \sout{protect herself from cold} \textbf{keep warm}.
  \item \sout{Ask help. Ask to help.} \textbf{Ask for help}.

  \item That would work as well.
  \item Exam: Pass an exam. Fail an exam. Do well/badly in an exam.
  \item Take the second right. Excuse me, please, could you (give me directions) /
        (tell me the way) to The London Eye?
  \item Не могли бы вы поставить (тарелку, например. За столом): Could you please, put it down.
  \item Ask something in caffee: Please, could I have a green tea? I would like a green tea, please.
  \item Она подсадила меня на YouTube: She has got me addicted to YouTube.
\end{ItemizeWithOrigParIndent}



%================================================================================================
\subsection{Pronunciation}
\begin{ItemizeWithOrigParIndent}
  \item \textbf{-ng}: ki\textbf{ng}, swimmi\textbf{ng}, includi\textbf{ng}.
  \item s\textbf{i}x: и.
  \item \textbf{h}im, \textbf{h}er, \textbf{h}is, \textbf{h}ave.
  \item \textbf{w}ere: \sout{в}[еэ]. \textbf{у[вф]}[еэ].
\end{ItemizeWithOrigParIndent}



%================================================================================================
\subsection{Vocabulary}

\subsubsection{New words, phrases}
\begin{description}[leftmargin=\OriginalParIndent,rightmargin=1mm, style=nextline, before={\renewcommand\makelabel[1]{##1~---}}]
  \item[Allotment] брит. небольшой участок земли, сдаваемый в аренду под огород.
  \item[Kettle] чайник.
  \item[Teabag] чайный пакетик.
  \item[Hedge] живая изгородь.
  \item[Hen] курица.
  \item[Gravy] подливка (из сока жаркого), соус, сок.
  \item[Admit one] ticket for one.
  \item[Stroll around] wander.
  \item[Afterward, beforehand] не надо ставить после них существительного (в отличие от after, before),
                             например: \textit{What will you do afterwards?}
  \item[Riddle] загадка, тайна.
  \item[Tame (squirrels)] приручённый, одомашненный; укрощённый (о животных).
  \item[Wages] зарплата.
  \item[Renovate] обновленная кухня, реконструированный парк~--- \textit{renovated kitchen}, \textit{renovated park}.
  \item[Do a chore] делать повседневную работу по дому.

  \item[Flat] квартира в многоквартирном доме.
  \item[Apartment building] многоквартирные дом.
  \item[House] частный дом.

  \item[Go/come] TODO.
  \item[Take/bring] TODO.
\end{description}

\subsubsection{Phrases verbs}
\begin{description}[leftmargin=2.2cm,style=nextline,before={\renewcommand\makelabel[1]{##1~---}}]
  \item[Call off] cancel (a meeting).
  \item[Use up] израсходовать.
  \item[Clean out] ``обчистить''.
  \item[Put on] надеть.
  \item[Take off] раздеться.
  \item[Make up] invent, выдумать.
  \item[Leave out] skip, пропускать, не принимать во внимание.
  \item[Lock in] запереть (себя внутри).
  \item[Bring up] поднимать вопрос (неприятный).
  \item[Hand out] раздавать.
  \item[Hand in] передавать (в руки).
  \item[Turn over] переворачивать.
\end{description}

\subsubsection{Saying, idioms}
\begin{ItemizeWithOrigParIndent}
  \item \ex{Bad hair day}~--- a day on which everything seems to go wrong.
  \item \ex{Can't stand / Can't bear}~--- means to be unable to tolerate somebody or something.
  \item \ex{Watched pot never boils}~---something we wait for with impatient attention seems to take forever.
  \item \ex{Jump on the bandwagon}~--- if you jump on the bandwagon, you join a growing movement in support
  of someone or something when that movement is seen to be about to become successful.
  \item \ex{The pot calling the kettle black}~---an idiom used to claim that a person is guilty of the very thing of which they accuse another.
\end{ItemizeWithOrigParIndent}

\subsubsection{Degree of likability}
\begin{itemize}
  \item Despise~--- презирать.
  \item Abhor/Detest~--- ненавидеть, питать отвращение; не выносить, не терпеть.
  \item Hate.
  \item Dislike/don't like.
  \item To be easy: \textit{I am easy} (informal).
  \item Do not mind~--- it's ok. To \textit{mind} something means, that you are distressed, annoyed or worried
  by it. Examples:
  \begin{itemize}
    \item \textit{Do you mind V\textsubscript{ing}? No, I don't mind.}
    \item \textit{Do you mind helping me with my homework? I don't mind helping you with\dots}
  \end{itemize}
  \item Quite like / quite enjoy.
  \item Like/enjoy.
  \item Love.
  \item Adore.
\end{itemize}
\vspace{-\parskip}
It seems that all these words must be followed by gerund or noun, see \ref{sed:Verbs_followed_by_gerunds}.


%================================================================================================
\subsection{Riddle}
\textit{Smith, where Jones had had ``had'', had had ``had had''. ``Had had'' had had examiner's approval.}




%================================================================================================
\subsection{Pronouns}
\textit{I don’t like vegetables, I don't like \sout{it} \textbf{them}.}



%================================================================================================
\subsection{Plural forms}
Nouns ending in -o: \textit{tomato\ex{es}}.\\
Irregular: \ex{foot~-- feet}, \ex{tooth~-- teeth}.\\
Does not change in plural: \ex{deer}, \ex{shrimp}, \ex{moose}, \ex{fish}.




%================================================================================================
\subsection{Expressing probability}
Might, could.\\
\ex{Might} not necessarily means past! \ex{Could} implies several choices: \textit{We could go to the cinema, or we could listen to music.}






%================================================================================================
\subsection{Articles}

\subsubsection{Fixed expressions}

\begin{minipage}{0.32\linewidth}
the sea\\
the environment\\
the country\\
the Internet\\
the cinema\\
the theatre\\
the opera\\
the radio\\
the navy\\
the army\\
the air force\\
the police
\end{minipage}
\begin{minipage}{0.65\linewidth}
the bank\\
the post office\\
the cinema\\
the station\\
the store\\
the doctor\\
the dentist\\
the airport: \textit{they are building a new airport} (a kind of airport)\textit{, because the
old airport is not big enough} (a specific one)
\end{minipage}

\begin{multicols}{3}
at the top\\
at the bottom\\
in the middle\\
at the centre\\
on the right/left\\
at the end
\end{multicols}

\begin{multicols}{3}
\za lunch\\
\za breakfast\\
\za dinner
\end{multicols}

\textit{He will go at \za school.} Если можно переформулировать без school, например: он пойдет учиться.\\
\textit{He will go at \textbf{the} school, where his parents work.} Если имеется в виду конкретное здание (specific place), а не процесс.\\
Вместо \ex{school} могут быть: \ex{hospital}, \ex{bed}, \ex{university}, \ex{jail}, \ex{prison}, \ex{college}.



\subsubsection{Buildings / Places.}
Places used \textbf{without the}:
\begin{description}[leftmargin=2.9cm,style=nextline,before={\renewcommand\makelabel[1]{##1:}}]
  \item[Streets, roads] \za Fleet Street.
  \item[Airports] \za Heathrow airport.
  \item[Parks] \za Hyde Park.
  \item[Squares] \za Times Square.
\end{description}
\vspace{-\parskip}
Name + building: \textit{\za Oxford University}, \textit{\za San Diego Zoo}, but \textit{\ex{The} Royal Academy} and \textit{\ex{The} Provincial Courts} because \textit{Royal} and \textit{Provincial} are adjectives.




\vspace{2mm}
Places used \textbf{with the}:
\begin{description}[leftmargin=3.9cm,rightmargin=\OriginalParIndent,style=nextline,before={\renewcommand\makelabel[1]{##1:}}]
  \item[Hotels] The Hilton Hotel.
  \item[Theaters, cinemas, operas] The Sydney Opera.
  \item[Museums] The British Museum.
  \item[Monuments, towers] The London Eye.
  \item[Names with ``of''] The University of Oxford. (But, Oxford University, see above.)
\end{description}
\vspace{-\parskip}
If any of those (restaurant, hotel, \dots) named after people, whose name ends with \ex{-s}, \ex{-s’} or \ex{-’s} do not use \ex{the}: \ex{\za St James’ Church}.



\subsubsection{Companies / Organizations.}
We do not use \ex{the} with names of companies:\\
\textit{\za Google}, \textit{\za Adidas}.

We often use \ex{the} with names of organizations and newspapers:
\textit{\ex{The} United Nations}, \textit{\ex{The} New York Times}.



\subsubsection{Geographical areas.}
We \textbf{do not use articles with}:

\hspace{\OriginalParIndent}\begin{minipage}{0.55\linewidth}
Cities, towns, villages.\\
Countries, provinces, states.\\
Continents.
\end{minipage}
\begin{minipage}{0.3\linewidth}
Islands.\\
Mountains.
\end{minipage}\vspace{\parskip}


We \textbf{use \ex{the} with}:

\hspace{\OriginalParIndent}\begin{minipage}{0.55\linewidth}
Oceans, seas.\\
Rivers, canals, straits.\\
Deserts.
\end{minipage}
\begin{minipage}{0.3\linewidth}
Groups of islands.\\
Mountain ranges.
\end{minipage}\vspace{\parskip}

We also \textbf{use \ex{the}} with compass directions: \textit{The North}, \textit{The North-East}, \textit{The East}, \dots: \textit{\ex{The} North of Germany}, \textit{\ex{The} East of Africa}, \dots\\
Do not mix up situations in which compass direction is part of name:
\textit{\za South America} (name of continent), \dots\\
But \textbf{do not use \ex{the}} with \textit{Southern}, \textit{Northern}, \dots







%================================================================================================
\subsection{Prepositions}

\begin{ItemizeWithOrigParIndent}
  \item Mr Jones is not here now. He is \ex{at} lunch.
  \item Tony is waiting for you \ex{at} the restaurant.
  \item The speep are \ex{in} the field.
  \item Hens are \ex{in} the garden.
\end{ItemizeWithOrigParIndent}

\begin{ItemizeWithOrigParIndent}
  \item \ex{On} train.
  \item \ex{On} the underground.
  \item \ex{By} bus.
  \item \ex{On} foot.
\end{ItemizeWithOrigParIndent}






%================================================================================================
\subsection{That/this/those/these.}
\textit{\ex{That} milk in the fridge is more than two weeks old.} \ex{That} is used because milk in
the fridge~--- fridge is closed.

\subsubsection{That's ok}
\begin{ItemizeWithOrigParIndent}
  \item Sorry about \ex{that}.
  \item \ex{That} was a terrible film. (But \textit{\ex{This} is a terrible film}~---
  \textit{is}~--- we are watching it now).
  \item I didn't know \ex{that}.
  \item \ex{That's} wonderful.
\end{ItemizeWithOrigParIndent}





%================================================================================================
\subsection{Linking words}
\subsubsection{Besides}
As a preposition \ex{besides} has the same meaning as ``in addition to'' and ``as well as''.\\
\textit{\ex{Besides} being a magnificent chief, he also paint extremely well.}\\
\textit{\ex{Besides} the clarinet, she also plays the flute.}

Do not mix:\\
Beside~--- next to, near, close to.\\
Besides~--- in addition to.

Let's pretend you had a long day at work, and you are tired. You just want to stay at home and watch TV. Your friend calls and asks ``do you want to join me for a dinner?''.
Your answer: \textit{No, thanks. Today was a long day and I'm tired. \ex{Besides}, I just ate dinner.}





%================================================================================================
\subsection{Relative clauses}

Correction bad/incorrect relative clauses:
\begin{ItemizeWithOrigParIndent}
  \item The tree \sout{whose} leaves have fallen off, is really old --> The tree \textbf{that has lost} all its
  leaves, is really old.
  \item I prefer to eat at restaurant \sout{whose} tables aren't very close together. --> I prefer to eat at restaurant \textbf{with tables, that} are further apart. / I prefer to eat at restaurant \textbf{where the tables} are further apart.
  \item That is the door, \sout{whose} lock doesn't work. --> That is the door \textbf{with lock, that} doesn't work.
\end{ItemizeWithOrigParIndent}





%================================================================================================
\subsection{Verbs followed by gerunds only.} \label{sed:Verbs_followed_by_gerunds}
These verbs are commonly followed by gerunds:
\vspace{-4\parskip}
\begin{multicols}{4}
\begin{itemize}[leftmargin=3.05mm]
  \item postpone
  \item delay
  \item finish
  \item feel like
  \item enjoy
  \item tolerate
  \item imagine
  \item can't help
  \item involve
  \item avoid
  \item keep
  \item anticipate
  \item complete
  \item admit
  \item report
  \item defend
  \item mention
  \item deny
  \item suggest
  \item discuss
  \item recall
  \item consider
  \item recommend
  \item risk
  \item resist
  \item mind
  \item practice
  \item despise
  \item dislike
  \item appreciate
  \item quit
\end{itemize}
\end{multicols}
\vspace{-5\parskip}
\textit{Doctor \ex{recommended} \sout{you} \ex{eating} healthy food.}\\
\ex{Feel like} is always followed by a gerund.\\
To \textit{\ex{feel like} doing something} means to be in the mood to do something.\\
\ex{can't help doing}~--- не мочь удержаться от чего-либо.\\
\textit{He \ex{admitted} stealing money from her pursue}~--- Он признался в краже денег из её кошелька.








%================================================================================================
\subsection{Wishes.}

\begin{tabular}{l@{\hspace{5.5mm}} l@{\hspace{4mm}} l}
\hline
future  & \textit{would + inf}   & \multirow{2}{*}{\textit{could + inf}}        \\
present & past simple   &                                     \\
\hline
past    & past perfect  & \textit{could + have + V\textsubscript{3}}   \\
\hline
\end{tabular}

\textbf{Notes:}

\textit{could + have + V\textsubscript{3}}~--- often means disappointment, and definitely impossible: \textit{I wish I could have gone to the party}.

\textit{wish \dots wouldn't \dots}~--- for complains about something that happens over and over again: \textit{I wish out neighbours wouldn't play that terrible music}.

\textbf{Be careful with future!}\\
It is not actually future.
\textit{would} in table above is \textit{past(will)}, where ''will`` means \textbf{``show willingness''}. That's why following sentences are incorrect:
\begin{ItemizeWithOrigParIndent}
  \item \textit{\sout{I wish I would know her.}} Дословный перевод: \textit{Я желаю что бы я зажелал её узнать}.
  \item \textit{\sout{I wish I would have a new phone.}} Дословный перевод: \textit{Я желаю что бы я зажелал заиметь новый телефон}.
\end{ItemizeWithOrigParIndent}
\dots and should be fixed as follows:
\begin{ItemizeWithOrigParIndent}
  \item \textit{I wish I knew her.}
  \item \textit{I wish I had a new phone.}
\end{ItemizeWithOrigParIndent}

Where ``will'' means a future event, we cannot use ``wish'' and must use ``hope'':
\begin{ItemizeWithOrigParIndent}
  \item \textit{There's a strike tomorrow. I hope some buses will still be running.}
  \item \textit{I hope everything will be fine in your new job.}
\end{ItemizeWithOrigParIndent}



\end{document}
